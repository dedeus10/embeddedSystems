This is the quick start guide for the \mbox{\hyperlink{group__serial__group}{Serial Interface module}}, with step-\/by-\/step instructions on how to configure and use the serial in a selection of use cases.

The use cases contain several code fragments. The code fragments in the steps for setup can be copied into a custom initialization function, while the steps for usage can be copied into, e.\+g., the main application function.\hypertarget{serial_quickstart_serial_use_cases}{}\section{Serial use cases}\label{serial_quickstart_serial_use_cases}

\begin{DoxyItemize}
\item \mbox{\hyperlink{serial_quickstart_serial_basic_use_case}{Basic use case -\/ transmit a character}}
\item \mbox{\hyperlink{serial_use_case_1}{Advanced use case -\/ Send a packet of serial data}}
\end{DoxyItemize}\hypertarget{serial_quickstart_serial_basic_use_case}{}\section{Basic use case -\/ transmit a character}\label{serial_quickstart_serial_basic_use_case}
In this use case, the serial module is configured for\+:
\begin{DoxyItemize}
\item Using U\+S\+A\+R\+T\+D0
\item Baudrate\+: 9600
\item Character length\+: 8 bit
\item Parity mode\+: Disabled
\item Stop bit\+: None
\item R\+S232 mode
\end{DoxyItemize}

The use case waits for a received character on the configured U\+S\+A\+RT and echoes the character back to the same U\+S\+A\+RT.\hypertarget{serial_quickstart_serial_basic_use_case_setup}{}\section{Setup steps}\label{serial_quickstart_serial_basic_use_case_setup}
\hypertarget{serial_quickstart_serial_basic_use_case_setup_prereq}{}\subsection{Prerequisites}\label{serial_quickstart_serial_basic_use_case_setup_prereq}

\begin{DoxyEnumerate}
\item System Clock Management (sysclk)
\end{DoxyEnumerate}\hypertarget{serial_quickstart_serial_basic_use_case_setup_code}{}\subsection{Example code}\label{serial_quickstart_serial_basic_use_case_setup_code}
The following configuration must be added to the project (typically to a conf\+\_\+uart\+\_\+serial.\+h file, but it can also be added to your main application file.)

\begin{DoxyNote}{Note}
The following takes S\+A\+M3X configuration for example, other devices have similar configuration, but their parameters may be different, refer to corresponding header files.
\end{DoxyNote}

\begin{DoxyCode}{0}
\DoxyCodeLine{\textcolor{preprocessor}{\#define USART\_SERIAL                     \&USARTD0}}
\DoxyCodeLine{\textcolor{preprocessor}{\#define USART\_SERIAL\_BAUDRATE            9600}}
\DoxyCodeLine{\textcolor{preprocessor}{\#define USART\_SERIAL\_CHAR\_LENGTH         US\_MR\_CHRL\_8\_BIT}}
\DoxyCodeLine{\textcolor{preprocessor}{\#define USART\_SERIAL\_PARITY              US\_MR\_PAR\_NO}}
\DoxyCodeLine{\textcolor{preprocessor}{\#define USART\_SERIAL\_STOP\_BIT            false}}
\end{DoxyCode}


A variable for the received byte must be added\+: 
\begin{DoxyCode}{0}
\DoxyCodeLine{uint8\_t received\_byte; }
\end{DoxyCode}


Add to application initialization\+: 
\begin{DoxyCode}{0}
\DoxyCodeLine{sysclk\_init();}
\DoxyCodeLine{}
\DoxyCodeLine{\textcolor{keyword}{static} usart\_serial\_options\_t usart\_options = \{}
\DoxyCodeLine{   .baudrate = USART\_SERIAL\_BAUDRATE,}
\DoxyCodeLine{   .charlength = USART\_SERIAL\_CHAR\_LENGTH,}
\DoxyCodeLine{   .paritytype = USART\_SERIAL\_PARITY,}
\DoxyCodeLine{   .stopbits = USART\_SERIAL\_STOP\_BIT}
\DoxyCodeLine{\};}
\DoxyCodeLine{}
\DoxyCodeLine{usart\_serial\_init(USART\_SERIAL, \&usart\_options);}
\end{DoxyCode}
\hypertarget{serial_quickstart_serial_basic_use_case_setup_flow}{}\subsection{Workflow}\label{serial_quickstart_serial_basic_use_case_setup_flow}

\begin{DoxyEnumerate}
\item Initialize system clock\+:
\begin{DoxyItemize}
\item 
\begin{DoxyCode}{0}
\DoxyCodeLine{sysclk\_init(); }
\end{DoxyCode}

\end{DoxyItemize}
\item Create serial U\+S\+A\+RT options struct\+:
\begin{DoxyItemize}
\item 
\begin{DoxyCode}{0}
\DoxyCodeLine{\textcolor{keyword}{static} usart\_serial\_options\_t usart\_options = \{}
\DoxyCodeLine{   .baudrate = USART\_SERIAL\_BAUDRATE,}
\DoxyCodeLine{   .charlength = USART\_SERIAL\_CHAR\_LENGTH,}
\DoxyCodeLine{   .paritytype = USART\_SERIAL\_PARITY,}
\DoxyCodeLine{   .stopbits = USART\_SERIAL\_STOP\_BIT}
\DoxyCodeLine{\};}
\end{DoxyCode}

\end{DoxyItemize}
\item Initialize the serial service\+:
\begin{DoxyItemize}
\item 
\begin{DoxyCode}{0}
\DoxyCodeLine{usart\_serial\_init(USART\_SERIAL, \&usart\_options);}
\end{DoxyCode}

\end{DoxyItemize}
\end{DoxyEnumerate}\hypertarget{serial_quickstart_serial_basic_use_case_usage}{}\section{Usage steps}\label{serial_quickstart_serial_basic_use_case_usage}
\hypertarget{serial_quickstart_serial_basic_use_case_usage_code}{}\subsection{Example code}\label{serial_quickstart_serial_basic_use_case_usage_code}
Add to application C-\/file\+: 
\begin{DoxyCode}{0}
\DoxyCodeLine{usart\_serial\_getchar(USART\_SERIAL, \&received\_byte);}
\DoxyCodeLine{usart\_serial\_putchar(USART\_SERIAL, received\_byte);}
\end{DoxyCode}
\hypertarget{serial_quickstart_serial_basic_use_case_usage_flow}{}\subsection{Workflow}\label{serial_quickstart_serial_basic_use_case_usage_flow}

\begin{DoxyEnumerate}
\item Wait for reception of a character\+:
\begin{DoxyItemize}
\item 
\begin{DoxyCode}{0}
\DoxyCodeLine{usart\_serial\_getchar(USART\_SERIAL, \&received\_byte); }
\end{DoxyCode}

\end{DoxyItemize}
\item Echo the character back\+:
\begin{DoxyItemize}
\item 
\begin{DoxyCode}{0}
\DoxyCodeLine{usart\_serial\_putchar(USART\_SERIAL, received\_byte); }
\end{DoxyCode}
 
\end{DoxyItemize}
\end{DoxyEnumerate}\hypertarget{serial_use_case_1}{}\section{Advanced use case -\/ Send a packet of serial data}\label{serial_use_case_1}
In this use case, the U\+S\+A\+RT module is configured for\+:
\begin{DoxyItemize}
\item Using U\+S\+A\+R\+T\+D0
\item Baudrate\+: 9600
\item Character length\+: 8 bit
\item Parity mode\+: Disabled
\item Stop bit\+: None
\item R\+S232 mode
\end{DoxyItemize}

The use case sends a string of text through the U\+S\+A\+RT.\hypertarget{serial_use_case_1_serial_use_case_1_setup}{}\subsection{Setup steps}\label{serial_use_case_1_serial_use_case_1_setup}
\hypertarget{serial_use_case_1_serial_use_case_1_setup_prereq}{}\subsubsection{Prerequisites}\label{serial_use_case_1_serial_use_case_1_setup_prereq}

\begin{DoxyEnumerate}
\item System Clock Management (sysclk)
\end{DoxyEnumerate}\hypertarget{serial_use_case_1_serial_use_case_1_setup_code}{}\subsubsection{Example code}\label{serial_use_case_1_serial_use_case_1_setup_code}
The following configuration must be added to the project (typically to a conf\+\_\+uart\+\_\+serial.\+h file, but it can also be added to your main application file.)\+:

\begin{DoxyNote}{Note}
The following takes S\+A\+M3X configuration for example, other devices have similar configuration, but their parameters may be different, refer to corresponding header files.
\end{DoxyNote}

\begin{DoxyCode}{0}
\DoxyCodeLine{\textcolor{preprocessor}{\#define USART\_SERIAL                     \&USARTD0}}
\DoxyCodeLine{\textcolor{preprocessor}{\#define USART\_SERIAL\_BAUDRATE            9600}}
\DoxyCodeLine{\textcolor{preprocessor}{\#define USART\_SERIAL\_CHAR\_LENGTH         US\_MR\_CHRL\_8\_BIT}}
\DoxyCodeLine{\textcolor{preprocessor}{\#define USART\_SERIAL\_PARITY              US\_MR\_PAR\_NO}}
\DoxyCodeLine{\textcolor{preprocessor}{\#define USART\_SERIAL\_STOP\_BIT            false}}
\end{DoxyCode}


Add to application initialization\+: 
\begin{DoxyCode}{0}
\DoxyCodeLine{sysclk\_init();}
\DoxyCodeLine{}
\DoxyCodeLine{\textcolor{keyword}{static} usart\_serial\_options\_t usart\_options = \{}
\DoxyCodeLine{   .baudrate = USART\_SERIAL\_BAUDRATE,}
\DoxyCodeLine{   .charlength = USART\_SERIAL\_CHAR\_LENGTH,}
\DoxyCodeLine{   .paritytype = USART\_SERIAL\_PARITY,}
\DoxyCodeLine{   .stopbits = USART\_SERIAL\_STOP\_BIT}
\DoxyCodeLine{\};}
\DoxyCodeLine{}
\DoxyCodeLine{usart\_serial\_init(USART\_SERIAL, \&usart\_options);}
\end{DoxyCode}
\hypertarget{serial_use_case_1_serial_use_case_1_setup_flow}{}\subsubsection{Workflow}\label{serial_use_case_1_serial_use_case_1_setup_flow}

\begin{DoxyEnumerate}
\item Initialize system clock\+:
\begin{DoxyItemize}
\item 
\begin{DoxyCode}{0}
\DoxyCodeLine{sysclk\_init(); }
\end{DoxyCode}

\end{DoxyItemize}
\item Create U\+S\+A\+RT options struct\+:
\begin{DoxyItemize}
\item 
\begin{DoxyCode}{0}
\DoxyCodeLine{\textcolor{keyword}{static} usart\_serial\_options\_t usart\_options = \{}
\DoxyCodeLine{   .baudrate = USART\_SERIAL\_BAUDRATE,}
\DoxyCodeLine{   .charlength = USART\_SERIAL\_CHAR\_LENGTH,}
\DoxyCodeLine{   .paritytype = USART\_SERIAL\_PARITY,}
\DoxyCodeLine{   .stopbits = USART\_SERIAL\_STOP\_BIT}
\DoxyCodeLine{\};}
\end{DoxyCode}

\end{DoxyItemize}
\item Initialize in R\+S232 mode\+:
\begin{DoxyItemize}
\item 
\begin{DoxyCode}{0}
\DoxyCodeLine{usart\_serial\_init(USART\_SERIAL\_EXAMPLE, \&usart\_options); }
\end{DoxyCode}

\end{DoxyItemize}
\end{DoxyEnumerate}\hypertarget{serial_use_case_1_serial_use_case_1_usage}{}\subsection{Usage steps}\label{serial_use_case_1_serial_use_case_1_usage}
\hypertarget{serial_use_case_1_serial_use_case_1_usage_code}{}\subsubsection{Example code}\label{serial_use_case_1_serial_use_case_1_usage_code}
Add to, e.\+g., main loop in application C-\/file\+: 
\begin{DoxyCode}{0}
\DoxyCodeLine{usart\_serial\_write\_packet(USART\_SERIAL, \textcolor{stringliteral}{"Test String"}, strlen(\textcolor{stringliteral}{"Test String"}));}
\end{DoxyCode}
\hypertarget{serial_use_case_1_serial_use_case_1_usage_flow}{}\subsubsection{Workflow}\label{serial_use_case_1_serial_use_case_1_usage_flow}

\begin{DoxyEnumerate}
\item Write a string of text to the U\+S\+A\+RT\+:
\begin{DoxyItemize}
\item 
\begin{DoxyCode}{0}
\DoxyCodeLine{usart\_serial\_write\_packet(USART\_SERIAL, \textcolor{stringliteral}{"Test String"}, strlen(\textcolor{stringliteral}{"Test String"})); }
\end{DoxyCode}
 
\end{DoxyItemize}
\end{DoxyEnumerate}