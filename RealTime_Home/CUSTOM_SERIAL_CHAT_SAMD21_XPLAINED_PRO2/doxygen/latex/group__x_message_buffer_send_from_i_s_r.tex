\hypertarget{group__x_message_buffer_send_from_i_s_r}{}\section{x\+Message\+Buffer\+Send\+From\+I\+SR}
\label{group__x_message_buffer_send_from_i_s_r}\index{xMessageBufferSendFromISR@{xMessageBufferSendFromISR}}
\mbox{\hyperlink{message__buffer_8h}{message\+\_\+buffer.\+h}}


\begin{DoxyPre}
size\_t xMessageBufferSendFromISR( MessageBufferHandle\_t xMessageBuffer,
                                  const void *pvTxData,
                                  size\_t xDataLengthBytes,
                                  BaseType\_t *pxHigherPriorityTaskWoken );

\begin{DoxyPre}\end{DoxyPre}
\end{DoxyPre}



\begin{DoxyPre}
\begin{DoxyPre}   Interrupt safe version of the API function that sends a discrete message to
   the message buffer.  The message can be any length that fits within the
   buffer's free space, and is copied into the buffer.\end{DoxyPre}
\end{DoxyPre}



\begin{DoxyPre}
\begin{DoxyPre}   ***NOTE***:  Uniquely among FreeRTOS objects, the stream buffer
   implementation (so also the message buffer implementation, as message buffers
   are built on top of stream buffers) assumes there is only one task or
   interrupt that will write to the buffer (the writer), and only one task or
   interrupt that will read from the buffer (the reader).  It is safe for the
   writer and reader to be different tasks or interrupts, but, unlike other
   FreeRTOS objects, it is not safe to have multiple different writers or
   multiple different readers.  If there are to be multiple different writers
   then the application writer must place each call to a writing API function
   (such as \mbox{\hyperlink{message__buffer_8h_a858f6da6fe24a226c45caf1634ea1605}{xMessageBufferSend()}}) inside a critical section and set the send
   block time to 0.  Likewise, if there are to be multiple different readers
   then the application writer must place each call to a reading API function
   (such as xMessageBufferRead()) inside a critical section and set the receive
   block time to 0.\end{DoxyPre}
\end{DoxyPre}



\begin{DoxyPre}
\begin{DoxyPre}   Use \mbox{\hyperlink{message__buffer_8h_a858f6da6fe24a226c45caf1634ea1605}{xMessageBufferSend()}} to write to a message buffer from a task.  Use
   \mbox{\hyperlink{message__buffer_8h_aeef5b0c4f8c2db6ca2230a8874813e79}{xMessageBufferSendFromISR()}} to write to a message buffer from an interrupt
   service routine (ISR).\end{DoxyPre}
\end{DoxyPre}



\begin{DoxyPre}
\begin{DoxyPre}
\begin{DoxyParams}{Parameters}
{\em xMessageBuffer} & The handle of the message buffer to which a message is
   being sent.\\
\hline
{\em pvTxData} & A pointer to the message that is to be copied into the
   message buffer.\\
\hline
{\em xDataLengthBytes} & The length of the message.  That is, the number of
   bytes to copy from pvTxData into the message buffer.  When a message is
   written to the message buffer an additional sizeof( size\_t ) bytes are also
   written to store the message's length.  sizeof( size\_t ) is typically 4 bytes
   on a 32-bit architecture, so on most 32-bit architecture setting
   xDataLengthBytes to 20 will reduce the free space in the message buffer by 24
   bytes (20 bytes of message data and 4 bytes to hold the message length).\\
\hline
{\em pxHigherPriorityTaskWoken} & It is possible that a message buffer will
   have a task blocked on it waiting for data.  Calling
   \mbox{\hyperlink{message__buffer_8h_aeef5b0c4f8c2db6ca2230a8874813e79}{xMessageBufferSendFromISR()}} can make data available, and so cause a task that
   was waiting for data to leave the Blocked state.  If calling
   \mbox{\hyperlink{message__buffer_8h_aeef5b0c4f8c2db6ca2230a8874813e79}{xMessageBufferSendFromISR()}} causes a task to leave the Blocked state, and the
   unblocked task has a priority higher than the currently executing task (the
   task that was interrupted), then, internally, \mbox{\hyperlink{message__buffer_8h_aeef5b0c4f8c2db6ca2230a8874813e79}{xMessageBufferSendFromISR()}}
   will set *pxHigherPriorityTaskWoken to pdTRUE.  If
   \mbox{\hyperlink{message__buffer_8h_aeef5b0c4f8c2db6ca2230a8874813e79}{xMessageBufferSendFromISR()}} sets this value to pdTRUE, then normally a
   context switch should be performed before the interrupt is exited.  This will
   ensure that the interrupt returns directly to the highest priority Ready
   state task.  *pxHigherPriorityTaskWoken should be set to pdFALSE before it
   is passed into the function.  See the code example below for an example.\\
\hline
\end{DoxyParams}
\begin{DoxyReturn}{Returns}
The number of bytes actually written to the message buffer.  If the
   message buffer didn't have enough free space for the message to be stored
   then 0 is returned, otherwise xDataLengthBytes is returned.
\end{DoxyReturn}
Example use:

\begin{DoxyPre}
// A message buffer that has already been created.
MessageBufferHandle\_t xMessageBuffer;\end{DoxyPre}
\end{DoxyPre}
\end{DoxyPre}



\begin{DoxyPre}
\begin{DoxyPre}
\begin{DoxyPre}void vAnInterruptServiceRoutine( void )
\{
size\_t xBytesSent;
char *pcStringToSend = "String to send";
BaseType\_t xHigherPriorityTaskWoken = pdFALSE; // Initialised to pdFALSE.\end{DoxyPre}
\end{DoxyPre}
\end{DoxyPre}



\begin{DoxyPre}
\begin{DoxyPre}
\begin{DoxyPre}    // Attempt to send the string to the message buffer.
    xBytesSent = xMessageBufferSendFromISR( xMessageBuffer,
                                            ( void * ) pcStringToSend,
                                            strlen( pcStringToSend ),
                                            \&xHigherPriorityTaskWoken );\end{DoxyPre}
\end{DoxyPre}
\end{DoxyPre}



\begin{DoxyPre}
\begin{DoxyPre}
\begin{DoxyPre}    if( xBytesSent != strlen( pcStringToSend ) )
    \{
        // The string could not be added to the message buffer because there was
        // not enough free space in the buffer.
    \}\end{DoxyPre}
\end{DoxyPre}
\end{DoxyPre}



\begin{DoxyPre}
\begin{DoxyPre}
\begin{DoxyPre}    // If xHigherPriorityTaskWoken was set to pdTRUE inside
    // \mbox{\hyperlink{message__buffer_8h_aeef5b0c4f8c2db6ca2230a8874813e79}{xMessageBufferSendFromISR()}} then a task that has a priority above the
    // priority of the currently executing task was unblocked and a context
    // switch should be performed to ensure the ISR returns to the unblocked
    // task.  In most FreeRTOS ports this is done by simply passing
    // xHigherPriorityTaskWoken into taskYIELD\_FROM\_ISR(), which will test the
    // variables value, and perform the context switch if necessary.  Check the
    // documentation for the port in use for port specific instructions.
    taskYIELD\_FROM\_ISR( xHigherPriorityTaskWoken );
\}
\end{DoxyPre}
 \end{DoxyPre}
\end{DoxyPre}
