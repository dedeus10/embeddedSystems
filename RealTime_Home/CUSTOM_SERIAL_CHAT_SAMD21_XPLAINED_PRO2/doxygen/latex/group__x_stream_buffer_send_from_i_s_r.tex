\hypertarget{group__x_stream_buffer_send_from_i_s_r}{}\section{x\+Stream\+Buffer\+Send\+From\+I\+SR}
\label{group__x_stream_buffer_send_from_i_s_r}\index{xStreamBufferSendFromISR@{xStreamBufferSendFromISR}}
\mbox{\hyperlink{stream__buffer_8h}{stream\+\_\+buffer.\+h}}


\begin{DoxyPre}
size\_t xStreamBufferSendFromISR( StreamBufferHandle\_t xStreamBuffer,
                                 const void *pvTxData,
                                 size\_t xDataLengthBytes,
                                 BaseType\_t *pxHigherPriorityTaskWoken );

\begin{DoxyPre}\end{DoxyPre}
\end{DoxyPre}



\begin{DoxyPre}
\begin{DoxyPre}   Interrupt safe version of the API function that sends a stream of bytes to
   the stream buffer.\end{DoxyPre}
\end{DoxyPre}



\begin{DoxyPre}
\begin{DoxyPre}   ***NOTE***:  Uniquely among FreeRTOS objects, the stream buffer
   implementation (so also the message buffer implementation, as message buffers
   are built on top of stream buffers) assumes there is only one task or
   interrupt that will write to the buffer (the writer), and only one task or
   interrupt that will read from the buffer (the reader).  It is safe for the
   writer and reader to be different tasks or interrupts, but, unlike other
   FreeRTOS objects, it is not safe to have multiple different writers or
   multiple different readers.  If there are to be multiple different writers
   then the application writer must place each call to a writing API function
   (such as \mbox{\hyperlink{stream__buffer_8h_a35cdf3b6bf677086b9128782f762499d}{xStreamBufferSend()}}) inside a critical section and set the send
   block time to 0.  Likewise, if there are to be multiple different readers
   then the application writer must place each call to a reading API function
   (such as xStreamBufferRead()) inside a critical section and set the receive
   block time to 0.\end{DoxyPre}
\end{DoxyPre}



\begin{DoxyPre}
\begin{DoxyPre}   Use \mbox{\hyperlink{stream__buffer_8h_a35cdf3b6bf677086b9128782f762499d}{xStreamBufferSend()}} to write to a stream buffer from a task.  Use
   \mbox{\hyperlink{stream__buffer_8h_a1dab226e99230e01e79bc2b5c0605e44}{xStreamBufferSendFromISR()}} to write to a stream buffer from an interrupt
   service routine (ISR).\end{DoxyPre}
\end{DoxyPre}



\begin{DoxyPre}
\begin{DoxyPre}
\begin{DoxyParams}{Parameters}
{\em xStreamBuffer} & The handle of the stream buffer to which a stream is
   being sent.\\
\hline
{\em pvTxData} & A pointer to the data that is to be copied into the stream
   buffer.\\
\hline
{\em xDataLengthBytes} & The maximum number of bytes to copy from pvTxData
   into the stream buffer.\\
\hline
{\em pxHigherPriorityTaskWoken} & It is possible that a stream buffer will
   have a task blocked on it waiting for data.  Calling
   \mbox{\hyperlink{stream__buffer_8h_a1dab226e99230e01e79bc2b5c0605e44}{xStreamBufferSendFromISR()}} can make data available, and so cause a task that
   was waiting for data to leave the Blocked state.  If calling
   \mbox{\hyperlink{stream__buffer_8h_a1dab226e99230e01e79bc2b5c0605e44}{xStreamBufferSendFromISR()}} causes a task to leave the Blocked state, and the
   unblocked task has a priority higher than the currently executing task (the
   task that was interrupted), then, internally, \mbox{\hyperlink{stream__buffer_8h_a1dab226e99230e01e79bc2b5c0605e44}{xStreamBufferSendFromISR()}}
   will set *pxHigherPriorityTaskWoken to pdTRUE.  If
   \mbox{\hyperlink{stream__buffer_8h_a1dab226e99230e01e79bc2b5c0605e44}{xStreamBufferSendFromISR()}} sets this value to pdTRUE, then normally a
   context switch should be performed before the interrupt is exited.  This will
   ensure that the interrupt returns directly to the highest priority Ready
   state task.  *pxHigherPriorityTaskWoken should be set to pdFALSE before it
   is passed into the function.  See the example code below for an example.\\
\hline
\end{DoxyParams}
\begin{DoxyReturn}{Returns}
The number of bytes actually written to the stream buffer, which will
   be less than xDataLengthBytes if the stream buffer didn't have enough free
   space for all the bytes to be written.
\end{DoxyReturn}
Example use:

\begin{DoxyPre}
// A stream buffer that has already been created.
StreamBufferHandle\_t xStreamBuffer;\end{DoxyPre}
\end{DoxyPre}
\end{DoxyPre}



\begin{DoxyPre}
\begin{DoxyPre}
\begin{DoxyPre}void vAnInterruptServiceRoutine( void )
\{
size\_t xBytesSent;
char *pcStringToSend = "String to send";
BaseType\_t xHigherPriorityTaskWoken = pdFALSE; // Initialised to pdFALSE.\end{DoxyPre}
\end{DoxyPre}
\end{DoxyPre}



\begin{DoxyPre}
\begin{DoxyPre}
\begin{DoxyPre}    // Attempt to send the string to the stream buffer.
    xBytesSent = xStreamBufferSendFromISR( xStreamBuffer,
                                           ( void * ) pcStringToSend,
                                           strlen( pcStringToSend ),
                                           \&xHigherPriorityTaskWoken );\end{DoxyPre}
\end{DoxyPre}
\end{DoxyPre}



\begin{DoxyPre}
\begin{DoxyPre}
\begin{DoxyPre}    if( xBytesSent != strlen( pcStringToSend ) )
    \{
        // There was not enough free space in the stream buffer for the entire
        // string to be written, ut xBytesSent bytes were written.
    \}\end{DoxyPre}
\end{DoxyPre}
\end{DoxyPre}



\begin{DoxyPre}
\begin{DoxyPre}
\begin{DoxyPre}    // If xHigherPriorityTaskWoken was set to pdTRUE inside
    // \mbox{\hyperlink{stream__buffer_8h_a1dab226e99230e01e79bc2b5c0605e44}{xStreamBufferSendFromISR()}} then a task that has a priority above the
    // priority of the currently executing task was unblocked and a context
    // switch should be performed to ensure the ISR returns to the unblocked
    // task.  In most FreeRTOS ports this is done by simply passing
    // xHigherPriorityTaskWoken into taskYIELD\_FROM\_ISR(), which will test the
    // variables value, and perform the context switch if necessary.  Check the
    // documentation for the port in use for port specific instructions.
    taskYIELD\_FROM\_ISR( xHigherPriorityTaskWoken );
\}
\end{DoxyPre}
 \end{DoxyPre}
\end{DoxyPre}
